% !TeX root = main-minted-german.tex

\section{UML-Diagramme mit PlantUML}

Falls \lualatex{} verwendet wird und PlantUML installiert ist, können UML-Diagramme mittels PlantUML erstellt werden.

\ifluatex
  \IfFileExists{plantuml.sty}{\autoref{fig:umlcar} zeigt ein einfaches UML-Diagramm mittels PlantUML.

\begin{figure}
  \centering
  \begin{plantuml}
  @startuml
    class Car

    Driver - Car : drives >
    Car *- Wheel : have 4 >
    Car -- Person : < owns
  @enduml
  \end{plantuml}
  \caption{Einfaches PlantUML-Diagramm}
  \label{fig:umlcar}
\end{figure}

\autoref{fig:plantuml} zeigt das Beispiel von \autoref{fig:uml} mittels PlantUML.

\begin{figure}
  \centering
  \begin{plantuml}
    @startuml
    package "p" #DDDDDD {
      package "sp1" #DDDDDD {
        class A<T> {
          + n : uint
          + t : float
        }
        class B {
          + d : double
          - setB(b: B): void
          + getB(): B
        }
        note bottom of Class: A note on class B
      }
      package "sp2" #DDDDDD {
        interface  C << interface >> {
          + n : uint
          + s : string
        }
      }
      class D {
        + n : uint
      }

      sp2 ..> sp1 : N1
      note on link: An annotation

      B <|--D
      B "1 toto" --> "0..* tata" C
      D "1" o-- D : tutu
      D *--> "titi 0..*" C : << vector >>
    }
    @enduml
  \end{plantuml}
  \caption{PlantUML-Diagramm}
  \label{fig:plantuml}
\end{figure}
}{}
\fi

\section{minted}

\href{https://github.com/gpoore/minted}{minted} ist eine Alternative zum \href{https://ctan.org/pkg/listings}{lstlistings}-Paket, das erweiterte Syntax-Hervorhebungen erlaubt.
\autoref{lst:xml} zeigt ein XML-Listing.
% Cleveref and minted do not play together, so we have to use \lineref - seee https://tex.stackexchange.com/q/132420/9075
Man kann auch direkt auf eine Zeile verweisen: \lref{commentline}.

% Unbedingt Listing mit großem "l" benutzen, da sonst das Listing nicht im "Verzeichnis der Listings" auftaucht.
\begin{Listing}[hb]
  \begin{minted}[linenos=true,escapeinside=||]{xml}
<listing name="third sample">
  <!-- comment -->
  <content>not interesting</content> |\Vlabel{commentline}|
</listing>
\end{minted}
  \caption{XML-Dokument gerendert mittels minted}
  \label{lst:xml}
\end{Listing}
